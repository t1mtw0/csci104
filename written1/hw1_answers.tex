% preamble %
\documentclass{article}
\usepackage{amsmath}
\usepackage{amssymb}
\usepackage{graphicx}
\usepackage{epstopdf}
\usepackage{inputenc}
\usepackage{enumitem}
\usepackage{pagecolor}
\usepackage{pdflscape}
\usepackage{lmodern}
\usepackage{blindtext}
\usepackage{geometry}
\geometry{left=2cm,right=2cm,top=.5cm,bottom=2cm}
% end preamble %

\begin{document}
\pagenumbering{gobble}
\pagecolor{white}

\begin{sloppypar}

\title{CSCI104 Written Homework 1}

\author{Timothy Tso}

\maketitle

\begin{enumerate}
    \item Problem 1
    \begin{enumerate}[label=(\alph*)]
        \item Observe that after $k$ iterations of the loop we have $i=2^{2^k}$. Since the body of the loop only runs when $i<n$, we have that for some $n$, the loop runs

        \[ \lfloor \log_{2} (\log_{2} (n-1)) \rfloor \]

        times. Therefore the runtime is $\lfloor \log_{2} (\log_{2} (n-1)) \rfloor \cdot \Theta(1) = \Theta(\log (\log (n)))$.
        \item Observe that for some $n$, the inner for loop runs $n/\lfloor \sqrt{n} \rfloor$ times. Since the inner loop scales at a factor of $n^3$, the runtime of the inner loop is $\Theta(n^3)$. All together, the runtime is

        \[ n/\lfloor \sqrt{n} \rfloor \cdot \Theta(n^3) = \Theta(n^{3.5}) \]
        \item Observe that the innermost for loop will run at most $n$ times because the array $A$ is never changed. Since the inner loop runs at $\log_{2}(n)$, we have that the upper bound runtime is $O(n^2 + n\log(n)) = O(n^2)$. On the other hand, the lower bound runtime is also $\Omega(n^2)$ since the comparison has to run at least $n^2$ times, even if the inner loop never runs.
        \item Observe that the body of the if statement will run only if $n>10$. Since the smaller numbers of $n$ do not matter when considering runtime, we can assume $n>10$. Now observe that the body of the if statement will run

        \[ \lfloor \log_{\frac 3 2}(\frac n {10}) + 1 \rfloor \]

        times, and each time the body is run the number of operations is the current length of the array. Furthermore, the total number of operations coming from the $if$ body is $\Theta(n)$, so we can deduce that the runtime of the total function is $\Theta(n+n)=\Theta(n)$.
    \end{enumerate}
    \item Problem 2
    \begin{enumerate}[label=(\alph*)]
    \item We can list the recursive calls to $llrec$ as follows:
        
        \begin{itemize}
            \item $llrec({p1}, {p5})$ returns 1,5,2,6,3,4
            \item $llrec({p5}, {p2})$ returns 5,2,6,3,4
            \item $llrec({p2}, {p6})$ returns 2,6,3,4
            \item $llrec({p6}, {p3})$ returns 6,3,4
            \item $llrec({p3}, NIL)$ returns 3,4
        \end{itemize}
        \item Since $in1=nullptr$, there is no recursive call and $in2=2$ is returned.
    \end{enumerate}
\end{enumerate}

\end{sloppypar}

\end{document}